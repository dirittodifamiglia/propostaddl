\documentclass[twocolumn,12pt]{article}
\usepackage[T1]{fontenc}
\usepackage{enumitem}
\usepackage[italian]{babel}
\usepackage{todonotes} % only for TODO - remove at the end
\usepackage{background} % only for draft background logo
\backgroundsetup{
  position=current page.east,
  angle=-90,
  nodeanchor=east,
  vshift=-5mm,
  hshift=+20mm,
  opacity=1,
  scale=3,
  contents=Draft - Work In Progress
}

\begin{document}

% Set a two columns layout with a line inbetween
\setlength{\columnseprule}{0.4pt}

\noindent Il presente disegno di legge intende attuare una riforma del diritto di famiglia in pieno accordo con la convenzione di New York del 20 novembre 1989 ratificata dalla legge 27 maggio 1991, n. 176 e rispettosa dei diritti costituzionali sanciti dall'art.3 della costituzione italiana per tutte le parti coinvolte in una separazione familiare.
\bigbreak
La legge 8 febbraio 2006, n.54 non ha avuto successo nel tutelare, come forma di affido prevalente, il modello bigenitoriale a tempi paritetici (ove applicabile). Inoltre non sempre tempi paritetici sono attuabili, in mancanza dei quali si pone spesso la necessità di stabilire assegni di mantenimento che, allo stato attuale, non seguono linee guida precise che tutelino entrambe le figure genitoriali coinvolte nel processo separativo e non godono di un processo decisionale trasparente e chiaro.
\bigbreak
L'articolo 1 di questo disegno di legge propugna il principio del mantenimento diretto e dei tempi paritetici. Il principio dell'affidamento congiunto con tempi quanto più possibili vicini all'affidamento con tempi paritetici è già stato ampiamente documentato in letteratura\footnote{Braver, Sanford L.; Lamb, Michael E. (10 April 2018) "Shared Parenting After Parental Separation: The Views of 12 Experts" - Journal of Divorce \& Remarriage}. Affidi fisicamente condivisi sembrano infatti contribuire significativamente al benessere dei bambini coinvolti in separazioni familiari, con percentuali minori di disturbi psichiatrici, abuso di sostanze stupefacenti, rendimenti scolastici, migliori stato di salute fisica e relazioni interparentali. Nessuno studio conosciuto ad oggi ha evidenziato aspetti negativi derivanti da una custodia fisica congiunta. Mediante il principio, ove applicabile, del mantenimento diretto, si intende altresì eliminare una comprovata fonte di contenzioso e di contrasto fra le figure parentali coinvolte nella separazione, assicurando pari dignità, doveri e diritti degli ex coniugi nei confronti dei figli.
\bigbreak

L'ipotesi di un assegno perequativo, già inteso nella norma attuale come espediente residuale, è purtroppo ad oggi applicata ordinariamente dai tribunali; è infatti prassi comune prevedere un assegno mensile a carico dell'uno o dell'altro genitore. In casi di applicabilità del mantenimento diretto con tempi paritetici (con spese straordinarie proporzionali al reddito), tale soluzione è da preferire per una moltitudine di ragioni fra cui:
\smallbreak
\begin{enumerate}[topsep=0pt,itemsep=-1ex,partopsep=1ex,parsep=1ex]
\item risulta complesso (e sebbene non venga usualmente richiesto, sarebbe giusto) avere una rendicontazione di come l'assegno sia stato speso per la propria prole: discutibilmente questo può essere considerato un importante punto di contrasto e disaccordo fra gli ex coniugi. La mancata rendicontazione non è più accettabile, così come non è più accettabile che un solo genitore si occupi prevalentemente di vestiario o altre spese ordinarie privando l'altro genitore non solo della possibilità di fare effettivamente il genitore spendendo tempo con i propri figli e scegliendo cosa acquistare, ma anche della possibilità di rifiutare una spesa piuttosto che un altra in ragione di una possibile sensata scelta educativa genitoriale - fermo restando l'interesse superiore del minore e non l'abuso di tale decisione contravvenendo agli obblighi assistenziali verso la prole (ipotesi che comunque non può essere ritenuta la norma);
\item ridurrebbe la pressione burocratica ed il carico già gravoso sui tribunali italiani per quanto concerne le revisioni degli assegni di mantenimento per mutate condizioni e le difficoltà di una modifica a copertura potenzialmente retroattiva;
\item eviterebbe contrasti ed iniquità su tutti i punti non esplicitamente coperti dalla sentenza stessa ad esempio sul tempo e carburante necessario dall'uno o dall'altro genitore (di solito il non collocatario) per prendere e riportare i minori all'abitazione del collocatario;
\item garantirebbero una sicurezza economica al genitore non collocatario che, in attesa di sentenza o di revisione dell'assegno di mantenimento, non potrebbe avere certezza al riguardo del carico economico mensile che egli/ella dovrà sostenere e non è quindi in grado di impegnarsi finanziariamente per accendere ad esempio un mutuo per una propria casa;
\end{enumerate}

In presenza di tempi paritetici, la ripartizione delle spese, in virtù della possibilità di rendicontazione e approvazione genitoriale delle spese straordinarie (che determinano anche la maggiore differenza nello stile di vita fra un minore di famiglia benestante ed un minore di famiglia non benestante), deve in ogni caso preferire l'assenza di un assegno di mantenimento e la ripartizione delle spese straordinarie in base alla \emph{disponibilità finanziaria} (cfr. articolo 2).

\bigbreak

L'articolo 2 di questo disegno si occupa invece di stabilire delle linee guida per la eventuale (e da intendersi come evenienza eccezionale) necessità di corresponsione di un assegno di mantenimento.
\smallbreak
Nell'ipotesi \emph{eccezionale} in cui tempi ragionevolmente paritetici siano oggettivamente o logisticamente infattibili, il calcolo dell'assegno di mantenimento di un coniuge verso l'altro non deve più essere determinato in virtù del semplice reddito fiscale, come spesso avviene nei tribunali italiani, ma in base al corretto principio della \emph{disponibilità finanziaria}. Non solo il reddito fiscale dei coniugi non è un indicatore affidabile del tenore di vita della famiglia, ma le modalità di determinazione delle diverse categorie di reddito previste dalla legge (lavoro autonomo, lavoro dipendente, rendite finanziarie, etc.) sono eterogenee: per i lavoratori dipendenti spesso il reddito netto corrisponde alla ricchezza disponibile, ma nel caso del reddito di lavoro autonomo e nel reddito di impresa il reddito fiscale non è un indicatore affidabile della ricchezza disponibile e del tenore di vita. Infatti dall'importo del reddito fiscale, per determinare la cifra disponibile per la vita privata, bisogna dedurre i costi che non sono deducibili ai fini fiscali, ma che l'imprenditore o il professionista deve comunque sostenere per esercitare la propria attività. La presenza o meno di reddito sommerso (evasione fiscale) non può essere in nessun caso assunta e nell'ipotesi in cui vi sia il sospetto di redditi non dichiarati, il giudice deve disporre un accertamento fiscale. Consulenze tecniche a professionisti della determinazione della reale disponibilità finanziaria devono infatti diventare prassi comune nei casi in cui non sia possibile avere un diretto raffronto fra i redditi degli ex coniugi.
\smallbreak
Non è altresì più accettabile che l'assegno di mantenimento debba in ogni caso garantire ai figli lo stesso tenore vissuto in costanza di convivenza con entrambi i genitori in quanto dopo una separazione le spese di solito (tranne casi eccezionali) non diminuiscono ma aumentano: le piccole e grandi economie di scala che erano possibili in convivenza come le spese di riscaldamento delle abitazioni, il gas necessario ai fornelli per la preparazione dei viveri, le spese di carburante e manutenzione veicoli per i nuovi tragitti dei genitori separati per stare con i propri figli, determinano disponibilità economiche totalmente differenti anche in caso di invariate condizioni lavorative e di reddito fiscale. Le condizioni dei minori devono in ogni caso risultare dignitose anche dopo una separazione, ma così devono risultare anche le condizioni degli ex coniugi. Allo stesso modo non è più accettabile che gli assegni di mantenimento prescindano dalle esigenze primarie del genitore onerato in caso debba esso sostenere affitti o altre necessarie spese.
\smallbreak
Nessun organo collegiale, istituzionale o meno, può ritenersi umanamente esente da errori: la determinazione dell'assegno di mantenimento ed il processo decisionale (in altre parole l'algoritmo utilizzato) devono essere meticolosamente documentati e riportati nelle sentenze. La possibilità di generare ulteriori contenziosi e ricorsi con la verifica della correttezza ad opera di consulenti professionali di terze parti devono essere accettati nel nome della ricerca della soluzione più razionale ed eticamente rispettosa dei diritti dei minori e dei genitori coinvolti nel processo separativo.

\bigbreak
\todo[inline]{TODO: assegnazione della casa coniugale}
\todo[inline]{TODO: fine del mantenimento a 18-21 anni come tutti i paesi europei}
\todo[inline]{TODO: abrogazione della firma degli altri coniugi per rinnovo passaporto CONIUGE (non minori)}

\clearpage

\begin{center}
    \textbf{DISEGNO DI LEGGE}
    \\\noindent\rule{2cm}{0.4pt}
    \smallbreak
\end{center}

\begin{center}
    Art. 1
\end{center}

L'articolo 337-\emph{ter} del codice civile è sostituito dal seguente:
\bigbreak
\guillemotleft Art. 337-\emph{ter} - \emph{(Provvedimenti riguardo ai figli)} -
Indipendentemente dai rapporti intercorrenti tra i due genitori, il figlio minore, nel proprio esclusivo interesse morale e materiale, ha il diritto di mantenere un rapporto equilibrato e continuativo con il padre e con la madre, di ricevere cura, educazione, istruzione e assistenza morale da entrambe le figure genitoriali, con paritetica assunzione di responsabilità e di impegni e con pari opportunità. Ha anche il diritto di trascorrere con ciascuno dei genitori tempi paritetici o equipollenti, salvi i casi di impossibilità materiale.

Qualora uno dei genitori ne faccia richiesta e non sussistano oggettivi elementi ostativi, il giudice assicura con idoneo provvedimento il diritto del minore di trascorrere tempi paritetici in ragione della metà del proprio tempo, compresi i pernottamenti, con ciascuno dei genitori. Salvo diverso accordo tra le parti, deve in ogni caso essere garantita alla prole la permanenza di non meno di dodici giorni al mese, compresi i pernottamenti, presso il padre e presso la madre, salvo comprovato e motivato pericolo di pregiudizio per la salute psico-fisica del figlio minore in caso di:
\begin{enumerate}[topsep=0pt,itemsep=-1ex,partopsep=1ex,parsep=1ex]
\item violenza;
\item abuso sessuale;
\item trascuratezza;
\item indisponibilità di un genitore;
\item inadeguatezza evidente degli spazi predisposti per la vita del minore.
\end{enumerate}
\bigbreak
\indent Il giudice o le parti, quando le circostanze rendano difficile attuare una divisione paritaria dei tempi su base mensile, possono
prevedere adeguati meccanismi di recupero durante i periodi di vacanza, onde garantire una sostanziale equivalenza dei tempi di frequentazione del minore con ciascuno dei genitori nel corso dell’anno.

Il figlio minore ha inoltre il diritto di conservare rapporti significativi con gli ascendenti e con i parenti di ciascun ramo genitoriale. Gli ascendenti del minore possono intervenire nel giudizio di affidamento con le forme dell’articolo 105 del codice di procedura civile. Il giudice, nei procedimenti di cui all’articolo 337-bis, adotta i provvedimenti relativi alla prole con esclusivo riferimento all’interesse morale e materiale di essa.

Il giudice, salvo che ciò sia contrario al superiore interesse del minore, affida in via condivisa i figli minori a entrambi i genitori e prende atto, se non contrari all’interesse dei figli, degli accordi intervenuti tra i genitori. Stabilisce il doppio domicilio del minore presso l’abitazione di ciascuno dei genitori ai fini delle comunicazioni scolastiche, amministrative e relative alla salute.

Ognuno dei genitori provvede al mantenimento in forma diretta dei figli per quanto riguarda le spese ordinarie, garantendo al minore
un dignitoso stile di vita. Per quanto concerne le spese straordinarie, ognuno dei genitori contribuisce in misura proporzionale alla propria
capacità reddituale.

Il giudice stabilisce, ove strettamente necessario e solo in via residuale, la corresponsione a carico di uno dei genitori, di un
assegno periodico per un tempo determinato in favore dell’altro a titolo di contributo al mantenimento del figlio minore. Nel medesimo provvedimento deve anche indicare quali iniziative devono essere intraprese dalle parti per giungere al mantenimento diretto della prole, indicando infine i termini entro i quali la corresponsione di assegno periodico residuale verrà a cessare. I benefici previdenziali e fiscali erogati in favore della prole o ai genitori per i figli a carico sono in ogni caso attribuiti sulla base del reciproco accordo ovvero su disposizione del giudice in misura direttamente proporzionale ai rispettivi redditi.

Ove le informazioni di carattere economico fornite dai genitori non risultino sufficientemente documentate, il giudice dispone un accertamento della polizia tributaria sui redditi e sui beni oggetto della contestazione, anche se intestati a soggetti diversi.

All’attuazione dei provvedimenti relativi all’affidamento della prole provvede il giudice del merito. La responsabilità genitoriale
è esercitata da entrambi i genitori. Le decisioni quotidiane sono assunte dal genitore che in quel momento si trova col figlio minore, mentre quelle di maggiore interesse per i figli relative all’istruzione, all’educazione, alla salute e alla scelta della residenza
abituale del minore sono assunte di comune accordo tenendo conto delle capacità, dell’inclinazione naturale e delle aspirazioni dei
figli. In caso di disaccordo la decisione è rimessa al giudice. Qualora il genitore non si attenga alle condizioni dettate, il giudice valuta detto comportamento anche al fine della modifica della forma di affidamento
\guillemotright

\bigbreak
\todo[inline]{TODO: inserire qui Art. 16 da DDL735 - Poteri del giudice e ascolto del minore}
\todo[inline]{TODO: articolo su criteri di determinazione assegno di mantenimento e trasparenza nel processo decisionale che deve essere documentato e reso pubblico alle parti interessate}

\end{document}