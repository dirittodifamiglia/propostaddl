\documentclass[twocolumn,12pt]{article}
\usepackage[T1]{fontenc}
\usepackage{enumitem}
\usepackage{todonotes} % only for TODO - remove at the end

\begin{document}

% Set a two columns layout with a line inbetween
\setlength{\columnseprule}{0.4pt}

\noindent Il presente disegno di legge intende attuare una riforma del diritto di famiglia in pieno accordo con la convenzione di New York del 20 novembre 1989 ratificata dalla legge 27 maggio 1991, n. 176 e rispettosa dei diritti costituzionali sanciti dall'art.3 della costituzione italiana per tutte le parti coinvolte in una separazione familiare.
\bigbreak
La legge 8 febbraio 2006, n.54 non ha avuto successo nel tutelare, come forma di affido prevalente, il modello bigenitoriale a tempi paritetici (ove applicabile). Inoltre non sempre tempi paritetici sono attuabili, in mancanza dei quali si pone spesso la necessità di stabilire assegni di mantenimento che, allo stato attuale, non seguono linee guida precise che tutelino entrambe le figure genitoriali coinvolte nel processo separativo e non godono di un processo decisionale trasparente e chiaro.
\bigbreak
L'Art. 1 di questo disegno di legge pro\-pugna il principio del mantenimento diretto e dei tempi paritetici, principio già ampia\-mente documentato in letteratura\footnote{Braver, Sanford L.; Lamb, Michael E. (10 April 2018) "Shared Parenting After Parental Separation: The Views of 12 Experts" - Journal of Divorce \& Remarriage}. Affidi fisicamente condivisi sembrano infatti contribuire significativamente al benessere dei bambini coinvolti in separazioni familiari, con percentuali minori di disturbi psichiatrici, abuso di sostanze stupefacenti, rendimenti scolastici, migliori stato di salute fisica e relazioni interparentali. Nessuno studio conosciuto ad oggi ha evidenziato aspetti negativi derivanti da una custodia fisica congiunta. Mediante il principio, ove applicabile, del mantenimento diretto, si intende altresì eliminare una comprovata fonte di contenzioso e di contrasto fra le figure parentali coinvolte nella separazione, assicurando pari dignità, doveri e diritti degli ex coniugi nei confronti dei figli.
\bigbreak


\bigbreak
\todo[inline]{TODO: dichiarazione dei redditi non misura ricchezza disponibile}
\todo[inline]{TODO: necessità di trasparenza nei processi decisionali della determinazione dell'assegno di mantenimento, soprattutto perequativi}

\clearpage

\begin{center}
    \textbf{DISEGNO DI LEGGE}
    \\\noindent\rule{2cm}{0.4pt}
    \smallbreak
\end{center}

\begin{center}
    Art. 1
\end{center}

L'articolo 337-\emph{ter} del codice civile è sostituito dal seguente:
\bigbreak
\guillemotleft Art. 337-\emph{ter} - \emph{(Provvedimenti riguardo ai figli)} -
Indipendentemente dai rapporti
intercorrenti tra i due genitori, il figlio minore, nel proprio esclusivo interesse morale
e materiale, ha il diritto di mantenere un
rapporto equilibrato e continuativo con il padre e con la madre, di ricevere cura, educazione, istruzione e assistenza morale da entrambe le figure genitoriali, con paritetica
assunzione di responsabilità e di impegni e
con pari opportunità. Ha anche il diritto di
trascorrere con ciascuno dei genitori tempi
paritetici o equipollenti, salvi i casi di impossibilità materiale.

Qualora uno dei genitori ne faccia richiesta e non sussistano oggettivi elementi ostativi, il giudice assicura con idoneo provvedimento il diritto del minore di trascorrere
tempi paritetici in ragione della metà del
proprio tempo, compresi i pernottamenti,
con ciascuno dei genitori. Salvo diverso accordo tra le parti, deve in ogni caso essere
garantita alla prole la permanenza di non
meno di dodici giorni al mese, compresi i
pernottamenti, presso il padre e presso la
madre, salvo comprovato e motivato pericolo di pregiudizio per la salute psico-fisica
del figlio minore in caso di:
\begin{enumerate}[topsep=0pt,itemsep=-1ex,partopsep=1ex,parsep=1ex]
\item violenza;
\item abuso sessuale;
\item trascuratezza;
\item indisponibilità di un genitore;
\item inadeguatezza evidente degli spazi predisposti per la vita del minore.
\end{enumerate}
\bigbreak
\indent Il giudice o le parti, quando le circostanze
rendano difficile attuare una divisione paritaria dei tempi su base mensile, possono
prevedere adeguati meccanismi di recupero
durante i periodi di vacanza, onde garantire
una sostanziale equivalenza dei tempi di frequentazione del minore con ciascuno dei genitori nel corso dell’anno.

Il figlio minore ha inoltre il diritto di conservare rapporti significativi con gli ascendenti e con i parenti di ciascun ramo genitoriale. Gli ascendenti del minore possono
intervenire nel giudizio di affidamento con
le forme dell’articolo 105 del codice di procedura civile. Il giudice, nei procedimenti di
cui all’articolo 337-bis, adotta i provvedimenti relativi alla prole con esclusivo riferimento all’interesse morale e materiale di
essa.

Il giudice, salvo che ciò sia contrario al
superiore interesse del minore, affida in via
condivisa i figli minori a entrambi i genitori
e prende atto, se non contrari all’interesse
dei figli, degli accordi intervenuti tra i genitori. Stabilisce il doppio domicilio del minore presso l’abitazione di ciascuno dei ge-
nitori ai fini delle comunicazioni scolastiche,
amministrative e relative alla salute.

Ognuno dei genitori provvede al mantenimento in forma diretta
dei figli per quanto riguarda le spese ordinarie, garantendo al minore
un dignitoso stile di vita. Per quanto concerne le spese straordinarie,
ognuno dei genitori contribuisce in misura proporzionale alla propria
capacità reddituale.

Il giudice stabilisce, ove strettamente necessario e solo in via residuale, la corresponsione a carico di uno dei genitori, di un
assegno periodico per un tempo determinato
in favore dell’altro a titolo di contributo al
mantenimento del figlio minore. Nel medesimo provvedimento deve anche indicare
quali iniziative devono essere intraprese
dalle parti per giungere al mantenimento diretto della prole, indicando infine i termini
entro i quali la corresponsione di assegno
periodico residuale verrà a cessare. I benefici previdenziali e fiscali erogati in favore
della prole o ai genitori per i figli a carico
sono in ogni caso attribuiti sulla base del reciproco accordo ovvero su disposizione del
giudice in misura direttamente proporzionale
ai rispettivi redditi.

Ove le informazioni di
carattere economico fornite dai genitori non
risultino sufficientemente documentate, il
giudice dispone un accertamento della polizia tributaria sui redditi e sui beni oggetto
della contestazione, anche se intestati a soggetti diversi.

All’attuazione dei provvedimenti relativi
all’affidamento della prole provvede il giudice del merito. La responsabilità genitoriale
è esercitata da entrambi i genitori. Le decisioni quotidiane sono assunte dal genitore
che in quel momento si trova col figlio minore, mentre quelle di maggiore interesse
per i figli relative all’istruzione, all’educazione, alla salute e alla scelta della residenza
abituale del minore sono assunte di comune
accordo tenendo conto delle capacità, dell’inclinazione naturale e delle aspirazioni dei
figli. In caso di disaccordo la decisione è rimessa al giudice. Qualora il genitore non si
attenga alle condizioni dettate, il giudice valuta detto comportamento anche al fine della
modifica della forma di affidamento
\guillemotright

\bigbreak
\todo[inline]{TODO: inserire qui Art. 16 da DDL735 - Poteri del giudice e ascolto del minore}
\todo[inline]{TODO: articolo su criteri di determinazione assegno di mantenimento e trasparenza nel processo decisionale che deve essere documentato e reso pubblico alle parti interessate}

\end{document}